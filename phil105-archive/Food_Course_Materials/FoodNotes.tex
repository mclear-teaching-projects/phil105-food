% Created 2016-01-21 Thu 20:39
\documentclass[11pt]{article}
\usepackage[utf8]{inputenc}
\usepackage[T1]{fontenc}
\usepackage{fixltx2e}
\usepackage{graphicx}
\usepackage{longtable}
\usepackage{float}
\usepackage{wrapfig}
\usepackage{rotating}
\usepackage[normalem]{ulem}
\usepackage{amsmath}
\usepackage{textcomp}
\usepackage{marvosym}
\usepackage{wasysym}
\usepackage{amssymb}
\usepackage{hyperref}
\tolerance=1000
\author{Colin McLear}
\date{\today}
\title{Philosophy of Food (PHIL 105)}
\hypersetup{
  pdfkeywords={},
  pdfsubject={},
  pdfcreator={Emacs 24.5.1 (Org mode 8.2.10)}}
\begin{document}

\maketitle

\section*{Bulletin}
\label{sec-1}
A wide-ranging examination of the philosophical, political, social, and economic
aspects of food, its production and consumption. Topics include the ethical
treatment of animals, factory farming, food justice, the relation of food to
social and religious identity, and climate change.

\section*{Ace Requirements \& Outcomes}
\label{sec-2}
\subsection*{SLO5 Proposal}
\label{sec-2-1}
Use knowledge, historical perspectives, analysis, interpretation, critical
evaluation, and the standards of evidence appropriate to the humanities to
address problems and issues.

\begin{enumerate}
\item Describe opportunities students should have to learn the outcome. How is the learning objective embedded in the course?

\begin{quote}
The purpose of this class is to focus on a topic of central
interest---food---and the philosophical, political, and economic questions
that arise from its consideration. In every case, students will be engaged
in the use of analysis, interpretation, and critical evaluation to address
philosophical problems and issues. The class will focus on reasons for and
against adopting various conclusions on a wide range of issues, and will
explore the cogency of the reasons offered. Another goal of this course is
that students will work out their own positions and come to better
understand opposing views. In the course of examining these reasons, they
will become familiar with various philosophical approaches, and they will
explore how various positions are amenable to justification.
\end{quote}

\item Describe student work that will be used to assess student achievement of the outcome and explain how the students demonstrate the knowledge and skills specified by the outcome.

\begin{quote}
  Students will be asked to interpret, analyze, and critically evaluate
  philosophical views and arguments through a variety of instruments, but
  largely through two examinations and a paper assignment.

  For example, students will be required to write an anlaytical essay, on a
  suggested topic, defending a thesis. One such topic would be whether eating
  meat is morally permissible. The student might argue that the fact that an
  animal feels pain counts as a reason for not eating it, and defend this by
  means of appeal to a utilitarian moral framework that the student would then
  explain. The paper would assess the student's ability to critically reason,
  their grasp of philosophical theory (in the explication of a particular moral
  theory), and their comprehension of specific material read in class (e.g.
  Peter Singer's utilitarian argument for vegetarianism).

  The exams are a combination of argument extraction (the student reads a short
  text and indicates what, if any argument is given), multiple choice,
  definition, and short essay. The multiple choice and definition parts of the
  exam assess the student's grasp of material discussed in class and readings
  assigned. The short essay prompts concern the student's ability to explicate a
  complex concept (e.g. the concept of welfare and its significance for
  evaluating the industrial food system), or require the explication of a
  particular dialectic as discussed in class (e.g. summarizing arguments for and
  against the moral permissibly of eating meat).
\end{quote}

\item As part of the ACE certification process, the department/unit agrees to collect and assess a reasonable sample of students’ work and provide reflections on students’ achievement of the Learning Outcomes for its respective ACE-certified courses. Please comment on your plans to develop a process to collect and evaluate student work over time for the purpose of assessing student success for this ACE outcome.

\begin{quote}
For purposes of assessment, the department commits to collection and analyzing a reasonable sample of student work. The material will be reviewed by the appropriate department subcommittee, which will assess student achievement, and, when necessary, make recommendations for enhancing achievement. The department will archive the samples along with the results of the review and any recommendations.
\end{quote}
\end{enumerate}

\subsection*{SLO8 Proposal}
\label{sec-2-2}
Explain ethical principles, civics, and stewardship, and their importance to
society. 

\begin{enumerate}
\item Describe opportunities students should have to learn the outcome. How is the learning objective embedded in the course?

\begin{quote}
The purpose of this class is to focus on a topic of central
interest---food---and the philosophical, political, and economic questions
that arise from its consideration. In every case, students will be engaged
in the use of analysis, interpretation, and critical evaluation to address
philosophical problems and issues. Stewardship, i.e., managing or guiding
something, is implicit in any discussion of ethics, for it is by our ethical
principles that we manage and guide our own behavior and try to influence
the world around us. Stewardship will more explicitly discussed in the
context of animal rights, climate change, and the consideration of how best
to moral, religious, and cultural commitments. The class will examine the
social importance of ethics, civic responsibility, and stewardship,
including reasons for and against adopting various courses of action, and
will explore the cogency of the reasons offered. Students will work out their
own positions and come to better understand opposing views. In the course of
examining these reasons, they will become familiar with various
philosophical approaches, and they will explore how various positions are
amenable to justification.
\end{quote}

\item Describe student work that will be used to assess student achievement of the outcome and explain how the students demonstrate the knowledge and skills specified by the outcome.

\begin{quote}
The ability to reason critically and knowledgably and to apply ethical principles to ethical issues regarding socially important civic responsibilities and to ethical issues regarding stewardship of socially important values will be assessed through a variety of instruments, but largely through examinations or paper assignments.

Students are required to write one analytical essay, on a suggested topic, in
which they defend a particular thesis. Topics include the moral permissibility
of eating meat and the significance of animal welfare in the assessment of the
industrial food system. Students will have to explain and defend their views on
such topics by appeal to moral frameworks that we have discussed in class, such
as consequentialism and deontology. Students will be expected to take a
particular position on the issue (at least for the purposes of the paper
exercise) and defend using one or more of the moral frameworks.

Students will also be assessed by means of regular quizzes. Such quizzes are
short and will typically take the form of true/false or multiple choice formats.
They are aimed at assessing and enforcing reading comprehension and basic uptake
of concepts introduced in the course of the term. For example, sample questions
might be the following:

1. According to utilitarianism, a morally required action is one which

A. maximizes the utility of all concerned
B. everyone agrees is correct
C. maximizes the utility of the weakest
D. avoids harming anyone  

2. True or False: According to Pollan, the most important element of the
industrial food chain is corn. 

\end{quote}

\item As part of the ACE certification process, the department/unit agrees to collect and assess a reasonable sample of students’ work and provide reflections on students’ achievement of the Learning Outcomes for its respective ACE-certified courses. Please comment on your plans to develop a process to collect and evaluate student work over time for the purpose of assessing student success for this ACE outcome.

\begin{quote}
For purposes of assessment, the department commits to collection and analyzing a reasonable sample of student work. The material will be reviewed by the appropriate department subcommittee, which will assess student achievement, and, when necessary, make recommendations for enhancing achievement. The department will archive the samples along with the results of the review and any recommendations.
\end{quote}
\end{enumerate}
% Emacs 24.5.1 (Org mode 8.2.10)
\end{document}
%%% Local Variables:
%%% mode: latex
%%% TeX-master: t
%%% End:
