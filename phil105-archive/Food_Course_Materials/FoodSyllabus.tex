%!TEX program = xelatex
\documentclass[12pt]{article}

%Definitions
\def\myauthor{Colin McLear}
\def\mytitle{Syllabus}
\def\mycopyright{\myauthor}
\def\mykeywords{}
\def\mybibliostyle{plain}
\def\mybibliocommand{}
\def\mysubtitle{}
\def\myaffiliation{University of Nebraska–Lincoln}
\def\myaddress{Department of Philosophy}
\def\myemail{mclear@unl.edu}
\def\myweb{http://colinmclear.net}
\def\myphone{607 216 8718}
\def\myfax{607 255 8177 }
\def\myversion{}
\def\myrevision{}

\date{} % not used (revision control instead)
\def\mykeywords{Philosophy, Syllabus}

%%%------------------------------------------------------------------------
%%% Required style files
%%%------------------------------------------------------------------------
\usepackage{url,fancyhdr}
\usepackage{comment}
\usepackage{lastpage}
\usepackage{enumitem}
\usepackage{epigraph}
%%\usepackage{revnum} % for reverse-numbered publications (revnumerate environment) if needed.
\usepackage{termcal}
\usepackage{csquotes}
%% needed for xelatex to work
\usepackage{fontspec}
\usepackage{xunicode}
\usepackage{mdwlist}
\usepackage[margin=1.25in]{geometry}
\usepackage{setspace}
%% color for the links
\usepackage{xcolor}
\definecolor{darkblue}{rgb}{0.0,0.0,.5}
%% hyperlinks
\usepackage[xetex, 
  colorlinks=true,
  urlcolor=darkblue,
  linkcolor=darkblue,
  plainpages=false,
    pdfpagelabels,
    unicode=false,
    bookmarks=true,
    pdftitle={\mytitle},
  	pagebackref,
  	pdfauthor={\myauthor},
    breaklinks=true,
    ]{hyperref}

% No section numbers
\makeatletter
\renewcommand\@seccntformat[1]{}
\makeatother

%%Calendar Stuff
% Few useful commands (our classes always meet either on Monday and Wednesday 
% or on Tuesday and Thursday)

\newcommand{\MWClass}{%
\calday[Monday]{\classday} % Monday
\skipday % Tuesday (no class)
\calday[Wednesday]{\classday} % Wednesday
\skipday % Thursday (no class)
\skipday % Friday 
\skipday\skipday % weekend (no class)
}

  \newcommand{\TRClass}{%
  \skipday % Monday (no class)
  \calday[Tuesday]{\classday} % Tuesday
  \skipday % Wednesday (no class)
  \calday[Thursday]{\classday} % Thursday
  \skipday % Friday 
  \skipday\skipday % weekend (no class)
  }

\newcommand{\Holiday}[2]{%
\options{#1}{\noclassday}
\caltext{#1}{#2}
}

\renewcommand{\calprintclass} % no class number

%%%%%%%%%%%%%%%%%%%%%%%%%%%%%



\begin{document}

\setmainfont[Mapping={tex-text},Numbers={OldStyle},Ligatures={Common}]{Minion Pro}
\setsansfont[Mapping=tex-text,Colour=AA0000]{Optima}
\setmonofont[Mapping=tex-text,Scale=0.9]{Inconsolata} 


%%%------------------------------------------------------------------------
%%% Page layout
%%%------------------------------------------------------------------------
\pagestyle{fancy}
\renewcommand{\headrulewidth}{0pt}
\fancyhead{}
\fancyfoot{}
%\rhead{{\scriptsize\thepage}}
\rfoot{\footnotesize \thepage\ | \pageref{LastPage}}
\lfoot{\footnotesize{\today}}
\setlength{\epigraphwidth}{.5\textwidth}


\fancypagestyle{firststyle}
{
  \fancyhf{}
  % $if(cfoot)$\cfoot{$cfoot$}$else$\cfoot{\footnotesize \thepage\ | \pageref{LastPage}}$endif$
  \rfoot{\footnotesize \thepage\ | \pageref{LastPage}}
  \renewcommand{\headrulewidth}{0pt}
}
\thispagestyle{firststyle}
  %%%------------------------------------------------------------------------
  %%% Address and contact block
  %%%------------------------------------------------------------------------
  \begin{minipage}[t]{2.95in}
   \flushleft {\footnotesize\textsc{Instructor:} Colin McLear \\
  \textsc{Course:} PHIL 105 \\\textsc{Time:} T/R 9:30-10:20 \\\textsc{Location:} MORR 141\\\begin{singlespace}\textsc{Office:} 1003 Oldfather Hall\end{singlespace}}

  % M--F: 10:30-1:20 \\\textsc{Location:} OTHM--105\\\begin{singlespace}\textsc{Office:} 1003 Oldfather Hall\end{singlespace}} 
    
  \end{minipage}
  \hfill     
  %\begin{minipage}[t]{0.0in}
  % dummy (needed here)
  %\end{minipage}
  \hfill
  \begin{minipage}[t]{1.7in}
    \flushleft {\footnotesize  {\href{mailto:\myemail}{\myemail}}} \\
    {\footnotesize  {\href{\myweb}{colinmclear.net/phil-101}} \\\href{http://www.unl.edu/philosophy/}{UNL Philosophy}\\\begin{singlespace}\textsc{Office Hours:} T/R 11-12 \end{singlespace}}%\begin{singlespace}\textsc{Office Hours:} T/R 11-12 p.m.\end{singlespace}}
  \end{minipage}  

  \bigskip
  \bigskip

  %% Name 
  \noindent{\huge {\textsc{The Philosophy of Food}}}

  \bigskip

  \epigraph{First we eat, then we do everything else}{\textsc{M.F.K. Fisher}}
  \smallskip    

  \section*{Course Overview} 
  Food is a central part of human life, both in its production and consumption. Food is closely tied to the values that we hold, and the cultural identities that we endorse (e.g. the sorts of things that \emph{we} eat vs. the sorts of things that \emph{they} eat). Our choices about food, both as individuals and as a society raise a variety of moral, political, and economic questions. Some of the questions we shall pursue include:

  \begin{itemize}
  \item What \emph{is} food? Not everything we eat we can digest. Moreover, some
of the things we could eat and digest (such as other people), we don't. Why not?
  \item What are the major economic and political structures governing food
    production and comsumption in the United States and other western countries?
    How do these structures impact developing countries?
  \item What are the environmental and social consequences of various sorts of eating habits? For example, do food choices contribute to environmental degradation and social injustice?
  \item How should we treat the animals we eat? Do we have ethical obligations to treat them in particular ways?
  \item In what ways does food connect to religious and cultural identities? To
    what extent can a society legislate for or against food practices that
    impinge on such identities? 
  \end{itemize}

  \section*{Learning Outcomes} 
  This course satisfies ACE requirement 5, that students use knowledge,
  historical perspectives, analysis, interpretation, critical evaluation, and
  the standards of evidence appropriate to the humanities to address problems
  and issues. Students will be evaluated with respect to these outcomes by
  taking regular quizzes, and more comprehensivly, by two exams and a final paper.

  PHIL 105 also satisfies ACE requirement 8, that students use knowledge,
  theories, and analysis to explain ethical principles and their importance in
  society. Students will be evaluated with respect to these outcomes by
  answering essay questions in the two exams and by writing a paper.
 
  In completing this course students satisfy these outcomes by being able to (i)
  find the argument of a text and restate it clearly in their own words; (ii)
  explain viewpoints clearly that are not their own; (iii) think critically
  about the ideas discussed in this course, including the moral and political
  significance of our food choices; (iv) explain the practical significance of
  difference courses of action regarding our food choices, both as individuals
  and as a society; (v) write papers using theses, organization, arguments,
  evidence, and language suitable to analytical writing in general and the
  discipline of philosophy in particular.

  \section*{Evaluation}

  \textbf{One Essay}: 25\%
  \begin{itemize*}
  \item Explain and critically assess a philosophical argument. Topics will be provided. Approximately 3-4 pages.
  \end{itemize*}

  \noindent\textbf{Two Exams}: 45\%

  \begin{itemize*}
  \item The exams will involve a combination of short answer and short essay questions.
  \begin{itemize*}
  \item Mid-Term: 20\%
  \item Final: 25\%
  \end{itemize*}
  \end{itemize*}

  \noindent\textbf{Ten Quizzes}: 20\%
  \begin{itemize*}
  \item Brief review quizzes held during section. They will not be announced ahead of time. Your two lowest grades will be dropped and your highest counted twice. 
  \end{itemize*}

  \noindent\textbf{Participation Grade}: 10\%
  \begin{itemize}
  \item The participation grade takes into account your attendance in lecture and section as well as the quantity and quality of your participation.
  \end{itemize}
  \section*{Required Materials} Readings will be posted on the
    \href{http://colinmclear.net/phil105}{course website} under
    \href{http://colinmclear.net/phil105-assignments}{``Assignments''}. There is only one required book.
  \begin{itemize*}
    \item Pollan, Michael. \emph{The Omnivore's Dilemma: A Natural History of Four Meals}. The Penguin Press, 2007. ISBN: 9780143038580.
    \item iClicker: Instruction for registering and using your iClicker
      may be found on \href{http://colinmclear.net/phil105}{the course website} or at \href{http://its.unl.edu/srs}{http://its.unl.edu/srs}
  \end{itemize*}
  \textbf{Students are expected to bring all relevant materials to class.}

  \section*{Course Requirements}
  \begin{itemize}
  \item \textbf{Preparation}: You are expected to attend every class meeting fully prepared to discuss each assigned reading, to submit written work punctually, and to offer thoughtful and constructive responses to the remarks of your instructor and your classmates. Make sure that you bring the relevant readings with you to every lecture class. I further expect you to treat both the texts at hand and your classmates’ ideas with openness and respect. 
  \item \textbf{Attendance}: Attendance is required. You are also expected to attend every section meeting. 1/2 a letter grade will be deducted from your final course grade for every absence from section after your fifth. 
  \item \textbf{Website}: We will use a course website for all materials.
    The site is available at:\\
    \href{http://colinmclear.net/phil105}{http://colinmclear.net/phil105}. Upcoming assignments and readings will be posted there. Please let me know if you have any problems. Technical glitches, computer malfunctions and crashing hard drives are not excuses for failing to complete work in this class.
  \item \textbf{Format for Papers}: Please submit work as a .doc or .rtf file. All work must be typed. I will not accept any handwritten work aside from that we do in class. Your papers should be in 12-point Times New Roman font, double-spaced with margins set to one inch on all sides. Your name, my name, the date and assignment should appear in the top left hand corner of the first page. Your last name and page number must appear in the top right hand corner on each subsequent page. Please staple or paperclip hard copies of papers and drafts. You are responsible for the presentation of your papers.
  \item \textbf{Late Work}: Late papers and assignments will standardly be marked down by \textbf{1/3 of a letter grade for each day the work is late} (for example, from A- to B+, from B+ to B, and so on).
  \end{itemize}


  \section*{Policies}

  \begin{itemize}
  \item \textbf{Academic Integrity}: All the work you turn in (including papers, drafts, and discussion board posts) must be written by you specifically for this course. It must originate with you in form and content with all contributory sources fully and specifically acknowledged. Make yourself familiar with UNL's Student Code of Conduct and Academic Integrity Code, available \href{http://stuafs.unl.edu/ja/code/three.shtml}{online}. \textbf{In this course, the normal penalty for any violation of the code is an “F” for the semester}. Violations may have additional consequences including expulsion from the university. Don't plagiarize – It just isn't worth it.
  \item \textbf{University Policies}: This instructor respects and upholds University policies and regulations pertaining to the observation of religious holidays; assistance available to physically handicapped, visually and/or hearing impaired students; plagiarism; sexual harassment; and racial or ethnic discrimination. All students are advised to become familiar with the respective University regulations and are encouraged to bring any questions or concerns to the attention of the instructor. 
  \item \textbf{ADA}: In compliance with University policy and equal access laws, I am available to discuss appropriate academic accommodations that may be required for students with disabilities. Students are encouraged to register with Student Disability Services to verify their eligibility for appropriate accommodations.
  \item \textbf{Misc.}: Please turn off cell phones, beeping watches, and other gadgets that make noise before entering our classroom. Absolutely no texting is permitted during class. I will subtract up to five points from your participation grade each and every time your phone rings or I see you texting during class.
  \end{itemize}

  \section*{Further Resources}
  \begin{itemize}
  \item \textbf{Jargon}: It's important to be on top of the technical terms used by philosophers. Please ask for clarification of terms in class. You can also consult Jim Pryor's online \href{http://www.jimpryor.net/teaching/vocab/index.html}{“Philosophical Terms and Methods.”}
  \item \textbf{Writing a Philosophy paper}: Papers should adhere to some consistent practice of footnoting and citation (Chicago, MLA, etc.). I don't really mind which one you use as long as you are consistent. On writing a philosophy paper, there is no better on-line guide than \href{http://www.jimpryor.net/teaching/guidelines/writing.html}{Jim Pryor’s}. Please consult it. Hacker’s \href{http://www.amazon.com/Writers-Reference-Exercises-Diana-Hacker/dp/0312601476/ref=sr_1_1?ie=UTF8&qid=1374423680&sr=8-1&keywords=hacker+a+writer%27s+reference}{\emph{A Writer’s Reference}} is also extremely helpful. Useful online writing help may be found at the \href{http://owl.english.purdue.edu/owl/}{Purdue Online Writing Lab} at http://owl.english.purdue.edu/owl/.
  \item \textbf{Help with writing}: The University of Nebraska-Lincoln Writing Center can provide you with meaningful support as you write for this class as well as for every course in which you enroll. Trained peer consultants are available to talk with you as you plan, draft, and revise your writing. Please check \href{http://www.unl.edu/writing/}{the Writing Center website} for locations, hours, and information about scheduling appointments.

  \item \textbf{Reference}: The \href{http://plato.stanford.edu}{Stanford Encyclopedia of Philosophy} at http://plato.stanford.edu is an excellent online resource. 
  \end{itemize}


  %%%%%
  %Calandar
  %%%%%
  \paragraph*{\uppercase{Tentative Assignment Calendar}:}Please check the course assignments page online (\href{http://colinmclear.net/phil105-assignments.html}{http://colinmclear.net/phil105-assignments.html}) for the definitive schedule.
  \begin{center}

  \begin{calendar}{1/10/2017}{16} % Semester starts on 8/26/2013 and last for 17
  % weeks, including finals week

  \setlength{\calboxdepth}{.5in}
  \TRClass
  \caltexton{1}{\textbf{Introduction}\\What is food?}
  \caltextnext{The metaphysics of food, continued} 
  \caltextnext{Choose your food carefully: Food \& Cannabilism\\J. Wisnewski: “Murder, Cannibalism, and Indirect Suicide: A Philosophical Study of a Recent Case”\\J. Swift: “A Modest Proposal”}
  \caltextnext{Overview of the food system, with a focus on corn\\Michael Pollan, \emph{Omnivore’s Dilemma} (MP 1-84)}
  \caltextnext{Overview continued (MP 85-122)}
  \caltextnext{Overview continued\\Jonathan Safran Foer, \emph{Eating Animals},
  “Storytelling” (JSF 3-16) \& “All or Nothing or Something Else” (JSF 21-41)}
  \caltextnext{``Big Food''\\\emph{Food Inc} (film)\\Marion Nestle, \emph{Food Politics}, Ch.4-6}
  \caltextnext{The Individual in Food Politics\\Nestle, \emph{Food Politics}, Ch.7 } 
  \caltextnext{Catch up}  
  \caltextnext{The Ethics of Eating Animals\\  Jeff McMahan,
  ``\href{http://opinionator.blogs.nytimes.com/2010/09/19/the-meat-eaters/}{The
  Meat-Eaters},'' \emph{NYT} September 19, 2010\\David Foster Wallace,
  ``\href{http://www.gourmet.com/magazine/2000s/2004/08/consider_the_lobster}{Consider
  the Lobster},'' \emph{Gourmet}, August 2004.}
  \caltextnext{Consequentialism described\\James Rachels, \emph{Elements of Moral Philosophy}, chs. 7-8}
  \caltextnext{Consequentialism applied\\Alastair Norcross, ``Puppies, Pigs and People: Eating Meat and Marginal Cases''}
  \caltextnext{Deontology described: Immanuel Kant\\Immanuel Kant, ``Rational
  Beings Alone Have Moral Worth''\\Holly Wilson, ``The Green Kant: Kant's
  Treatment of Animals''}
  \caltextnext{Deontology applied\\Tom Regan, ``The Radical Egalitarian Case
  for Animal Rights''\\Mary Anne Warren, ``A Critique of Regan's Animal Rights
  Theory''}
  \caltextnext{Catch up}
  \caltextnext{Moral vegetarianism: a debate\\James Rachels, ``The Basic
  Argument for Vegetarianism''\\Michael Martin,
  ``\href{http://www.reasonpapers.com/pdf/03/rp_3_2.pdf}{A Critique of Moral
  Vegetarianism}''}
  \caltextnext{Ethically-acceptable meat?\\Pollan, \emph{The Omnivore's Dilemma}, Chapter 17 (MP 304-333)}
  \caltextnext{Ethically acceptable meat, continued\\Singer/Mason: \emph{The
  Ethics of What We Eat}, Chapter 17 (pp. 241-269)}
  \caltextnext{Mid-term Review}
  \caltextnext{\bf{Mid-Term Exam}} 
  \caltextnext{Food and Religious Identity\\Regan, ``Christians are what Christians Eat''\\Gaffney, ``Eastern Religions and the Eating of Meat''\\Preece, ``Ask your Brother for Forgiveness: Animal respect in Native
  American traditions''}
  \caltextnext{Religion, continued\\Regenstein et al., ``The Kosher and Halal Food Laws''\\Regenstein, ``The Politics of Religious Slaughter: How Science can be
  Misused''}
  \caltextnext{Feminist Perspectives\\Carol J. Adams, ``The Sexual Politics of Meat''\\Christina Van Dyke ``Gendered Eating'' from \emph{Philosophy Comes to
  Dinner,} Chignell, Cuneo, Halteman (eds.)}
  \caltextnext{Catch up}
  \caltextnext{Food Justice\\Norgaard et. al. ``A Continuing Legacy:
  institutional racism, hunger and nutritional justice on the Klamath''}
  \caltextnext{Justice, continued\\Watch: Alan Savory TED talk: \href{http://www.ted.com/talks/allan_savory_how_to_green_the_world_s_deserts_and_reverse_climate_change.html}{``How to fight desertification and fight climate change''}}
  \caltextnext{Environmental and Food Justice\\Mares \& Peña, ``Environmental
  and Food Justice: toward local, slow and deep food systems''\\Bill McKibben,
  ``A Special Moment in History: the challenge of overpopulation and
  overconsumption''}
  \caltextnext{Population and Justice\\Singer: ``Famine, Affluence, and Morality''}
  \caltextnext{Population and Justice, continued\\Hardin ``Lifeboat Ethics''\\Murdoch and
  Oaten ``Population and Food: a critique of life boat ethics''}

  % ... and so on

  % Holidays Fall

  \Holiday{11/27/2014}{\textbf{Thanksgiving Break -- No Class}}
  \Holiday{12/11/2014}{\textbf{Last Day of Class}\\Exam Review}

  % Holidays Spring

  \Holiday{3/22/2017}{\textbf{No Class --- Spring Break!}}
  \Holiday{3/24/2017}{\textbf{No Class --- Spring Break!}}
  \Holiday{4/28/2017}{\textbf{Last Day of Class}\\Exam Review}

  \end{calendar}
  \textbf{Final paper due by 4 p.m., Friday, April 29th}\\
  \textbf{Final exam: 10:00 to 12:00 noon, Tuesday, May 5th}

  \end{center}


  \end{document}
